\documentclass[11pt,oneside,a4paper, openany]{book}
\usepackage[usenames, dvipsnames]{xcolor}
\usepackage[utf8]{inputenc}
\usepackage[portuguese]{babel}
\usepackage[a4paper,top=3cm,bottom=2.3cm,left=2.2cm,right=2.2cm]{geometry} % margens
\usepackage{amsthm}
\usepackage{mathtools}
\usepackage{amssymb}
\usepackage{listings}
\usepackage{tikz}
\usepackage{tikz-qtree}
\usepackage{minted}
\usepackage{xspace}
\usepackage{subfig}
\usepackage{algorithm}
\usepackage[noend]{algpseudocode}
\usepackage{import}
\usepackage{hyperref}
\hypersetup{
    colorlinks,
    citecolor=violet!85!black,
    linkcolor=MidnightBlue
}
\usepackage{lmodern}

%\renewcommand*\thesection{\arabic{section}}

\renewcommand{\baselinestretch}{1.3}

%% BIBTEX
\usepackage{csquotes}
\usepackage[
backend=biber,
style=alphabetic,
sorting=nty,
maxbibnames=10
]{biblatex}
\addbibresource{ref.bib}
\renewbibmacro*{volume+number+eid}{%
  \printfield{volume}%
%  \setunit*{\adddot}% DELETED
  \setunit*{\addnbspace}% NEW (optional); there's also \addnbthinspace
  \printfield{number}%
  \setunit{\addcomma\space}%
  \printfield{eid}}
\DeclareFieldFormat[article]{number}{\mkbibparens{#1}}
%%%

\renewcommand\listingscaption{Código}

\definecolor{codegreen}{rgb}{0,0.6,0}
\definecolor{codegray}{rgb}{0.5,0.5,0.5}
\definecolor{codepurple}{rgb}{0.58,0,0.82}
\let\And\undefined
\let\Or\undefined

\algrenewcommand\algorithmicthen{:}
\algrenewcommand\algorithmicdo{:}
\algnewcommand\And{\textbf{ and }}
\algnewcommand\Or{\textbf{ or }}
\algnewcommand\New{\textbf{ new }}
\algnewcommand\To{\textbf{ to }}
\algnewcommand\DownTo{\textbf{ down to }}
\algnewcommand\Break{\textbf{break}}
\algnewcommand\Null{\textbf{null}}
\algnewcommand\True{\textbf{true}}
\makeatletter
 \renewcommand{\ALG@name}{Código}
\makeatother
\algdef{SE}[SUBALG]{Indent}{EndIndent}{}{\algorithmicend\ }%
\algtext*{Indent}
\algtext*{EndIndent}

\newcommand{\pluseq}{\mathbin{{\tiny +}{=}}}

\definecolor{mygreen}{rgb}{0,0.6,0}
\definecolor{mygray}{rgb}{0.5,0.5,0.5}
\definecolor{mymauve}{rgb}{0.58,0,0.82}

\renewcommand{\ttdefault}{pcr}

\newtheorem{theorem}{Teorema}[section]
\newtheorem*{complexity}{Complexidade}
\newtheorem{corollary}{Corolário}[theorem]
\newtheorem{lemma}[theorem]{Lema}
\newtheorem{prop}[theorem]{Proposição}
\newtheorem*{definition}{Definição}
\newtheorem*{invar}{Invariante}

\newcommand{\len}[1]{\mathit{len}(#1.s)}
\newcommand{\suf}{\mathit{suf}}
\newcommand{\ns}{\mathit{ns}}
\newcommand{\cd}{\mathit{cd}}
\newcommand{\cn}{\mathit{cn}}
\newcommand{\mmid}{\mathit{mid}}
\newcommand{\node}{\mathit{node}}
\newcommand{\mrk}{\mathit{mrk}}
%automata
\renewcommand{\root}{\mathit{root}}
\newcommand{\last}{\mathit{last}}
\newcommand{\Path}{\textsc{Path}}
\newcommand{\final}{\mathit{final}}
%manacher
\newcommand{\Center}{\textsc{Center}}
\newcommand{\Fill}{\textsc{Fill}}
\newcommand{\floor}[1]{\left \lfloor #1 \right \rfloor}
\newcommand{\ceil}[1]{\left \lceil #1 \right \rceil}

\newcommand{\cS}{\mathcal{S}}
\newcommand{\E}{\Sigma}
\newcommand{\keyword}[1]{\textbf{#1}}
\newcommand{\code}[1]{$#1$}
\newcommand{\Oh}{\mathcal{O}}
\newcommand{\pref}{\sqsubseteq}
\newcommand{\suff}{\sqsupseteq}
\newcommand{\prefp}{\sqsubset}
\newcommand{\suffp}{\sqsupset}
\newcommand{\pop}{\textsc{Pop}}
\newcommand{\tdots}{\,.\,.\,} % in place of \ldots
\newcommand{\func}[2]{\textsc{#1}$(#2)$}
\newcommand{\M}[1]{\mathcal{#1}}
\newcommand*{\nref}[1]{%
  {\protect\NoHyper\ref{#1}\protect\endNoHyper}%
}

%% AUTOMATON
\newcommand{\emptystring}{\varepsilon}
\newcommand{\stfail}{\eta}
\newcommand{\impl}{\Rightarrow}
%\newcommand{\iff}{\Leftrightarrow}
\usetikzlibrary{matrix,backgrounds, decorations.pathreplacing, automata, arrows}

%%%% COPIADO
\usepackage{setspace}                   % espaçamento flexível
\usepackage{indentfirst}                % indentação do primeiro parágrafo
\usepackage{type1cm}                    % fontes realmente escaláveis
\usepackage{titletoc}
\usepackage[font=small,format=plain,labelfont=bf]{caption}
\usepackage{courier}                    % usa o Adobe Courier no lugar de Computer Modern Typewriter


\fontsize{60}{62}\usefont{OT1}{cmr}{m}{n}{\selectfont}

\usepackage{fancyhdr}
\pagestyle{fancy}
\fancyhf{}
\renewcommand{\chaptermark}[1]{\markboth{\MakeUppercase{#1}}{}}
\renewcommand{\sectionmark}[1]{\markright{\MakeUppercase{#1}}{}}
\renewcommand{\subsectionmark}[1]{}
\renewcommand{\headrulewidth}{0pt}
\fancyhead[R]{{\footnotesize\rightmark}}
\fancyhead[L]{{\footnotesize\leftmark}}
\fancyfoot[C]{{\thepage}}
%\setcounter{tocdepth}{2}
\setlength{\headheight}{13.6pt} 
%%%%

\title{Algoritmos em sequências}
\author{Yan Couto}
\date{\today}

\sloppy

\begin{document}

%%%%%% COPIADO
\frontmatter 
% cabeçalho para as páginas das seções anteriores ao capítulo 1 (frontmatter)

\onehalfspacing  % espaçamento

% ---------------------------------------------------------------------------- %
% CAPA
% Nota: O título para as dissertações/teses do IME-USP devem caber em um 
% orifício de 10,7cm de largura x 6,0cm de altura que há na capa fornecida pela SPG.
\thispagestyle{empty}
\begin{center}
    \vspace*{2.3cm}
    \textbf{\huge{Algoritmos em sequências}}\\
    
    \vspace*{1cm}
    \Large{Yan Soares Couto}

    \vskip 1.8cm
    Orientadora: Cristina G. Fernandes\\

    \vspace{\fill}
    \normalsize{São Paulo, 2016}
\end{center}

% ---------------------------------------------------------------------------- %
% Resumo
\chapter*{Resumo}

\noindent COUTO, Y. S. \textbf{Algoritmos em sequências}. 
 Instituto de Matemática e Estatística,
Universidade de São Paulo, São Paulo, 2016.
\\

Algoritmos em sequências são úteis para buscas de padrões e em biologia computacional. É possível descobrir a maior substring palíndroma de uma string em tempo linear. Tries são estruturas para armazenamento de string, permitindo buscas rápidas por prefixos. Pode-se usar uma trie para buscar um conjunto de strings, como padrão, em uma sequência. A construção de uma trie para todos os sufixos de uma string pode ser feita em tempo linear. Pode-se construir um autômato que reconhece todos os sufixo de uma string em espaço e tempo linear.
\\

\noindent \textbf{Palavras-chave:} algoritmo, string, trie, autômato, palíndromo.


% ---------------------------------------------------------------------------- %
\pagenumbering{gobble}
% Sumário
\begingroup
\let\cleardoublepage\clearpage
\tableofcontents
\endgroup

% ---------------------------------------------------------------------------- %
% Capítulos do trabalho
\mainmatter

% cabeçalho para as páginas de todos os capítulos

%\singlespacing              % espaçamento simples
\onehalfspacing            % espaçamento um e meio
\pagenumbering{arabic}
%%%%%

\setcounter{secnumdepth}{0}


\chapter*{Introdução}
\chaptermark{Introdução}
\addcontentsline{toc}{chapter}{Introdução}


Encontrar todas as ocorrências de uma certa palavra em um texto é um problema muito importante, que vemos diariamente, por exemplo ao buscar uma palavra-chave em nossos emails recebidos ou em algum documento.

Nessa classe de problemas, de busca de ocorrências exatas, não são aceitas diferenças entre o padrão buscado e as posições de ocorrência retornadas. Outra classe de problemas é de busca de ocorrências inexatas, ou busca de ocorrências aproximadas, quando algumas diferenças são aceitas. Nesse caso é necessário adotar alguma definição de similaridade entre duas cadeias de caracteres, em geral o número mínimo de caracteres que precisam ser mudados para tornar ambas cadeias iguais. O utilitário \texttt{diff}, por exemplo, encontra o número mínimo de linhas que precisam ser mudadas para tornar dois arquivos iguais.

Outra área que se beneficia do desenvolvimento de algoritmos para tais problemas é a área da biologia computacional, pois DNA pode ser modelado como uma sequência de caracteres \texttt{A}, \texttt{T}, \texttt{C} e \texttt{G}. Assim, é possível encontrar similaridades entre trechos de DNAs diferentes, e buscar, em um conjunto enorme de trechos de DNA, algum similar ou idêntico a um trecho dado. Proteínas também podem ser modeladas como sequências de aminoácidos.

Os assuntos discutidos nesse trabalho são também muito úteis para praticantes de programação competitiva, pois os algoritmos abordados podem ser usados na resolução de vários problemas não triviais envolvendo strings, e têm implementação pequena.

\section{Objetivos e estrutura do texto}

Neste trabalho, apresentamos algoritmos e estruturas de dados relacionados a problemas de busca de ocorrências exatas. Os algoritmos clássicos para encontrar ocorrências exatas, como KMP e Boyer-Moore, não são abordados. O foco é em estruturas mais poderosas, que generalizam os algoritmos clássicos, são mais rápidas ou, quando mais complicadas, resolvem problemas mais difíceis.

Diferentemente dos livros e fontes mais comuns, a implementação é bastante discutida, com pseudocódigo usando apenas instruções presentes nas principais linguagens de programação. Porém, a teoria não é esquecida. Ao apresentar os algoritmos e suas explicações, a corretude e complexidade são provadas formalmente, em detalhes, de maneira tão completa quanto possível.

O Capítulo~\ref{chap:manacher} trata de um algoritmo útil para encontrar palíndromos em uma cadeia de caracteres, e é o tema que menos se relaciona à busca de ocorrências exatas, apesar de ser bastante similar a um algoritmo que serve justamente para busca de ocorrências exatas.

O Capítulo~\ref{chap:trie} apresenta uma estrutura de dados chamada trie, com a qual é possível armazenar um conjunto de strings de forma inteligente. Esta estrutura é usada nos dois capítulos seguintes de forma mais extensa.

O Capítulo~\ref{chap:aho} apresenta uma generalização de KMP que funciona com tries e não apenas strings, e dessa forma é possível buscar ocorrências de várias strings ao mesmo tempo.

O Capítulo~\ref{chap:suffixtree} mostra como construir uma trie para todos os sufixos de uma string, mas sem gastar tempo ou espaço proporcional ao tamanho de todos estes. Essa estrutura é extremamente poderosa, podendo de certa forma generalizar o algoritmo do Capítulo~\ref{chap:aho}.

O Capítulo~\ref{chap:suffixautomaton} apresenta um autômato que identifica todos os sufixos de uma string, e é tão poderoso quanto a estrutura do Capítulo~\ref{chap:suffixtree}.

As principais fontes usadas neste trabalho foram os livros de Gusfield~\cite{gusfield} e Crochemore, Hancart e Lecroq~\cite{crochemore}.

\section{Notação e definições básicas}

Uma \emph{string}~$T[1\tdots n]$ de tamanho~${|T| \coloneqq n}$ é um vetor de~$n$ elementos, onde todos elementos são de um alfabeto~$\E$ finito. Para facilitar a implementação, em geral assumimos que~${\E = \{0, \ldots, |\E| - 1\}}$, ou seja,~$\E$ tem uma ordem e seus elementos estão ``comprimidos'', mas nos exemplos usamos~${\E = \{\textnormal{a}, \ldots,  \textnormal{z}\}}$.

Dizemos que~${T[i\tdots j] \coloneqq T[i] T[i+1] \ldots T[j]}$ é uma \emph{substring} de~$T$. Se~${T_1 = T_2[1\tdots i]}$, para algum~$1 \leq i \leq |T_2|$, então~$T_1$ é \emph{prefixo} de~$T_2$, denotado por~$T_1 \pref T_2$. Da mesma forma, se~${T_1 = T_2[i\tdots |T_2|]}$ então~$T_1$ é \emph{sufixo} de~$T_2$, denotado por~${T_1 \suff T_2}$.
Um sufixo ou prefixo é \emph{próprio} se não é igual à string original, e nesse caso usamos~$\prefp$ e~$\suffp$. Ou seja,~${T_1 \prefp T_2}$ se e somente se ${T_1 \pref T_2}$ e ${T_1 \neq T_2}$.

Dizemos que $T_1$ ocorre em~$T_2$ se~$T_1$ é uma substring de~$T_2$. Existe uma ocorrência de~$T_1$ em~$T_2$ na posição~$i$ se~${T_1 \pref T_2[i\tdots |T_2|]}$, ou seja, se~${T_1 = T_2[i \tdots i + |T_1| - 1]}$.

\setcounter{secnumdepth}{3}


\import{/}{manacher.tex}

\import{/}{trie.tex}

\import{/}{aho.tex}

%==================================================================================================
%=====SSSSSSSSSS===================================================================================
%=====SSSSSSSSSS===================================================================================
%=====SSSS==========UUUU====UUUU===FFFFFFFFFF==FFFFFFFFFFF===III==XXXX=========XXXXX===============
%=====SSSS==========UUUU====UUUU===FFFFFFFFFF==FFFFFFFFFFF===III===XXXXX======XXXX=================
%=====SSSSSSSSSS    UUUU====UUUU===FFF=========FFF===========III=====XXXX====XXXX==================
%=====SSSSSSSSSS====UUUU====UUUU===FFFFFFF=====FFFFFFF=======III=======XXXX=XXX====================
%===========SSSS====UUUU====UUUU===FFFFFFF=====FFFFFFF=======III=========XXXX======================
%===========SSSS ===UUUU====UUUU===FFF=========FFF===========III========XX=XXX=====================
%=====SSSSSSSSSS====UUUUUUUUUUUU===FFF=========FFF===========III======XXX====XXX===================
%=====SSSSSSSSSS====UUUUUUUUUUUU===FFF=========FFF===========III====XXXX=======XXXX================
%==================================================================================================


%==================================================================================================
%=====TTTTTTTTTTTTTT===RRRRRRR========EEEEEEEEEE===EEEEEEEEEE======================================
%=====TTTTTTTTTTTTTT===RRR===RRR======EEEEEEEEEE===EEEEEEEEEE======================================
%=========TTTTT========RRR=====RRR====EEE==========EEE=============================================
%=========TTTTT========RRR====RR======EEE==========EEE=============================================
%=========TTTTT========RRRRRRRR=======EEEEEE=======EEEEEE==========================================
%=========TTTTT========RRR===RRR======EEEEEE=======EEEEEE==========================================
%=========TTTTT========RRR====RRRR====EEE==========EEE=============================================
%=========TTTTT========RRR=====RRR====EEEEEEEEE====EEEEEEEEE=======================================
%=========TTTTT========RRR=====RRR====EEEEEEEEE====EEEEEEEEE=======================================

\import{/}{tree.tex}


%==================================================================================================
%=====SSSSSSSSSS===================================================================================
%=====SSSSSSSSSS===================================================================================
%=====SSSS==========UUUU====UUUU===FFFFFFFFFF==FFFFFFFFFFF===III==XXXX=========XXXXX===============
%=====SSSS==========UUUU====UUUU===FFFFFFFFFF==FFFFFFFFFFF===III===XXXXX======XXXX=================
%=====SSSSSSSSSS    UUUU====UUUU===FFF=========FFF===========III=====XXXX====XXXX==================
%=====SSSSSSSSSS====UUUU====UUUU===FFFFFFF=====FFFFFFF=======III=======XXXX=XXX====================
%===========SSSS====UUUU====UUUU===FFFFFFF=====FFFFFFF=======III=========XXXX======================
%===========SSSS ===UUUU====UUUU===FFF=========FFF===========III========XX=XXX=====================
%=====SSSSSSSSSS====UUUUUUUUUUUU===FFF=========FFF===========III======XXX====XXX===================
%=====SSSSSSSSSS====UUUUUUUUUUUU===FFF=========FFF===========III====XXXX=======XXXX================
%==================================================================================================


%==================================================================================================
%=====AAAAAAAAAAAA=================================================================================
%=====AAAA===AAAAA===UUU===UUU==TTTTTTTTTTTTT===OOOOOOOOOOO========================================
%=====AAAA====AAAA===UUU===UUU==TTTTTTTTTTTTT===OOOOOOOOOOO========================================
%=====AAAAAAAAAAAA===UUU===UUU======TTTT========OOO=====OOO========================================
%=====AAAAAAAAAAAA===UUU===UUU======TTTT========OOO=====OOO========================================
%=====AAAA====AAAA===UUU===UUU======TTTT========OOO=====OOO========================================
%=====AAAA====AAAA===UUUUUUUUU======TTTT========OOOOOOOOOOO===...====...====...====================
%=====AAAA====AAAA===UUUUUUUUU======TTTT========OOOOOOOOOOO===...====...====...====================
%==================================================================================================

\import{/}{automaton.tex}

\import{/}{subjetiva.tex}

\newpage
\printbibliography

\end{document}